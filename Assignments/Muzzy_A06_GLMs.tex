\documentclass[]{article}
\usepackage{lmodern}
\usepackage{amssymb,amsmath}
\usepackage{ifxetex,ifluatex}
\usepackage{fixltx2e} % provides \textsubscript
\ifnum 0\ifxetex 1\fi\ifluatex 1\fi=0 % if pdftex
  \usepackage[T1]{fontenc}
  \usepackage[utf8]{inputenc}
\else % if luatex or xelatex
  \ifxetex
    \usepackage{mathspec}
  \else
    \usepackage{fontspec}
  \fi
  \defaultfontfeatures{Ligatures=TeX,Scale=MatchLowercase}
\fi
% use upquote if available, for straight quotes in verbatim environments
\IfFileExists{upquote.sty}{\usepackage{upquote}}{}
% use microtype if available
\IfFileExists{microtype.sty}{%
\usepackage{microtype}
\UseMicrotypeSet[protrusion]{basicmath} % disable protrusion for tt fonts
}{}
\usepackage[margin=2.54cm]{geometry}
\usepackage{hyperref}
\hypersetup{unicode=true,
            pdftitle={Assignment 6: Generalized Linear Models},
            pdfauthor={Laurie Muzzy},
            pdfborder={0 0 0},
            breaklinks=true}
\urlstyle{same}  % don't use monospace font for urls
\usepackage{color}
\usepackage{fancyvrb}
\newcommand{\VerbBar}{|}
\newcommand{\VERB}{\Verb[commandchars=\\\{\}]}
\DefineVerbatimEnvironment{Highlighting}{Verbatim}{commandchars=\\\{\}}
% Add ',fontsize=\small' for more characters per line
\usepackage{framed}
\definecolor{shadecolor}{RGB}{248,248,248}
\newenvironment{Shaded}{\begin{snugshade}}{\end{snugshade}}
\newcommand{\KeywordTok}[1]{\textcolor[rgb]{0.13,0.29,0.53}{\textbf{#1}}}
\newcommand{\DataTypeTok}[1]{\textcolor[rgb]{0.13,0.29,0.53}{#1}}
\newcommand{\DecValTok}[1]{\textcolor[rgb]{0.00,0.00,0.81}{#1}}
\newcommand{\BaseNTok}[1]{\textcolor[rgb]{0.00,0.00,0.81}{#1}}
\newcommand{\FloatTok}[1]{\textcolor[rgb]{0.00,0.00,0.81}{#1}}
\newcommand{\ConstantTok}[1]{\textcolor[rgb]{0.00,0.00,0.00}{#1}}
\newcommand{\CharTok}[1]{\textcolor[rgb]{0.31,0.60,0.02}{#1}}
\newcommand{\SpecialCharTok}[1]{\textcolor[rgb]{0.00,0.00,0.00}{#1}}
\newcommand{\StringTok}[1]{\textcolor[rgb]{0.31,0.60,0.02}{#1}}
\newcommand{\VerbatimStringTok}[1]{\textcolor[rgb]{0.31,0.60,0.02}{#1}}
\newcommand{\SpecialStringTok}[1]{\textcolor[rgb]{0.31,0.60,0.02}{#1}}
\newcommand{\ImportTok}[1]{#1}
\newcommand{\CommentTok}[1]{\textcolor[rgb]{0.56,0.35,0.01}{\textit{#1}}}
\newcommand{\DocumentationTok}[1]{\textcolor[rgb]{0.56,0.35,0.01}{\textbf{\textit{#1}}}}
\newcommand{\AnnotationTok}[1]{\textcolor[rgb]{0.56,0.35,0.01}{\textbf{\textit{#1}}}}
\newcommand{\CommentVarTok}[1]{\textcolor[rgb]{0.56,0.35,0.01}{\textbf{\textit{#1}}}}
\newcommand{\OtherTok}[1]{\textcolor[rgb]{0.56,0.35,0.01}{#1}}
\newcommand{\FunctionTok}[1]{\textcolor[rgb]{0.00,0.00,0.00}{#1}}
\newcommand{\VariableTok}[1]{\textcolor[rgb]{0.00,0.00,0.00}{#1}}
\newcommand{\ControlFlowTok}[1]{\textcolor[rgb]{0.13,0.29,0.53}{\textbf{#1}}}
\newcommand{\OperatorTok}[1]{\textcolor[rgb]{0.81,0.36,0.00}{\textbf{#1}}}
\newcommand{\BuiltInTok}[1]{#1}
\newcommand{\ExtensionTok}[1]{#1}
\newcommand{\PreprocessorTok}[1]{\textcolor[rgb]{0.56,0.35,0.01}{\textit{#1}}}
\newcommand{\AttributeTok}[1]{\textcolor[rgb]{0.77,0.63,0.00}{#1}}
\newcommand{\RegionMarkerTok}[1]{#1}
\newcommand{\InformationTok}[1]{\textcolor[rgb]{0.56,0.35,0.01}{\textbf{\textit{#1}}}}
\newcommand{\WarningTok}[1]{\textcolor[rgb]{0.56,0.35,0.01}{\textbf{\textit{#1}}}}
\newcommand{\AlertTok}[1]{\textcolor[rgb]{0.94,0.16,0.16}{#1}}
\newcommand{\ErrorTok}[1]{\textcolor[rgb]{0.64,0.00,0.00}{\textbf{#1}}}
\newcommand{\NormalTok}[1]{#1}
\usepackage{graphicx,grffile}
\makeatletter
\def\maxwidth{\ifdim\Gin@nat@width>\linewidth\linewidth\else\Gin@nat@width\fi}
\def\maxheight{\ifdim\Gin@nat@height>\textheight\textheight\else\Gin@nat@height\fi}
\makeatother
% Scale images if necessary, so that they will not overflow the page
% margins by default, and it is still possible to overwrite the defaults
% using explicit options in \includegraphics[width, height, ...]{}
\setkeys{Gin}{width=\maxwidth,height=\maxheight,keepaspectratio}
\IfFileExists{parskip.sty}{%
\usepackage{parskip}
}{% else
\setlength{\parindent}{0pt}
\setlength{\parskip}{6pt plus 2pt minus 1pt}
}
\setlength{\emergencystretch}{3em}  % prevent overfull lines
\providecommand{\tightlist}{%
  \setlength{\itemsep}{0pt}\setlength{\parskip}{0pt}}
\setcounter{secnumdepth}{0}
% Redefines (sub)paragraphs to behave more like sections
\ifx\paragraph\undefined\else
\let\oldparagraph\paragraph
\renewcommand{\paragraph}[1]{\oldparagraph{#1}\mbox{}}
\fi
\ifx\subparagraph\undefined\else
\let\oldsubparagraph\subparagraph
\renewcommand{\subparagraph}[1]{\oldsubparagraph{#1}\mbox{}}
\fi

%%% Use protect on footnotes to avoid problems with footnotes in titles
\let\rmarkdownfootnote\footnote%
\def\footnote{\protect\rmarkdownfootnote}

%%% Change title format to be more compact
\usepackage{titling}

% Create subtitle command for use in maketitle
\newcommand{\subtitle}[1]{
  \posttitle{
    \begin{center}\large#1\end{center}
    }
}

\setlength{\droptitle}{-2em}

  \title{Assignment 6: Generalized Linear Models}
    \pretitle{\vspace{\droptitle}\centering\huge}
  \posttitle{\par}
    \author{Laurie Muzzy}
    \preauthor{\centering\large\emph}
  \postauthor{\par}
    \date{}
    \predate{}\postdate{}
  

\begin{document}
\maketitle

\subsection{OVERVIEW}\label{overview}

This exercise accompanies the lessons in Environmental Data Analytics
(ENV872L) on generalized linear models.

\subsection{Directions}\label{directions}

\begin{enumerate}
\def\labelenumi{\arabic{enumi}.}
\tightlist
\item
  Change ``Student Name'' on line 3 (above) with your name.
\item
  Use the lesson as a guide. It contains code that can be modified to
  complete the assignment.
\item
  Work through the steps, \textbf{creating code and output} that fulfill
  each instruction.
\item
  Be sure to \textbf{answer the questions} in this assignment document.
  Space for your answers is provided in this document and is indicated
  by the ``\textgreater{}'' character. If you need a second paragraph be
  sure to start the first line with ``\textgreater{}''. You should
  notice that the answer is highlighted in green by RStudio.
\item
  When you have completed the assignment, \textbf{Knit} the text and
  code into a single PDF file. You will need to have the correct
  software installed to do this (see Software Installation Guide) Press
  the \texttt{Knit} button in the RStudio scripting panel. This will
  save the PDF output in your Assignments folder.
\item
  After Knitting, please submit the completed exercise (PDF file) to the
  dropbox in Sakai. Please add your last name into the file name (e.g.,
  ``Salk\_A06\_GLMs.pdf'') prior to submission.
\end{enumerate}

The completed exercise is due on Tuesday, 26 February, 2019 before class
begins.

\subsection{Set up your session}\label{set-up-your-session}

\begin{enumerate}
\def\labelenumi{\arabic{enumi}.}
\item
  Set up your session. Upload the EPA Ecotox dataset for Neonicotinoids
  and the NTL-LTER raw data file for chemistry/physics.
\item
  Build a ggplot theme and set it as your default theme.
\end{enumerate}

\begin{Shaded}
\begin{Highlighting}[]
\CommentTok{#1}
\KeywordTok{library}\NormalTok{(tidyverse)}
\end{Highlighting}
\end{Shaded}

\begin{verbatim}
## Warning: package 'tidyverse' was built under R version 3.4.2
\end{verbatim}

\begin{verbatim}
## -- Attaching packages ----------------------------------- tidyverse 1.2.1 --
\end{verbatim}

\begin{verbatim}
## v ggplot2 3.1.0       v purrr   0.3.0  
## v tibble  2.0.1       v dplyr   0.8.0.1
## v tidyr   0.8.2       v stringr 1.3.1  
## v readr   1.3.1       v forcats 0.3.0
\end{verbatim}

\begin{verbatim}
## Warning: package 'ggplot2' was built under R version 3.4.4
\end{verbatim}

\begin{verbatim}
## Warning: package 'tibble' was built under R version 3.4.4
\end{verbatim}

\begin{verbatim}
## Warning: package 'tidyr' was built under R version 3.4.4
\end{verbatim}

\begin{verbatim}
## Warning: package 'readr' was built under R version 3.4.4
\end{verbatim}

\begin{verbatim}
## Warning: package 'purrr' was built under R version 3.4.4
\end{verbatim}

\begin{verbatim}
## Warning: package 'dplyr' was built under R version 3.4.4
\end{verbatim}

\begin{verbatim}
## Warning: package 'stringr' was built under R version 3.4.4
\end{verbatim}

\begin{verbatim}
## Warning: package 'forcats' was built under R version 3.4.3
\end{verbatim}

\begin{verbatim}
## -- Conflicts -------------------------------------- tidyverse_conflicts() --
## x dplyr::filter() masks stats::filter()
## x dplyr::lag()    masks stats::lag()
\end{verbatim}

\begin{Shaded}
\begin{Highlighting}[]
\KeywordTok{getwd}\NormalTok{() }
\end{Highlighting}
\end{Shaded}

\begin{verbatim}
## [1] "/Users/laurie/Desktop/Envtl_Data_Analytics/MuzzyGitFile"
\end{verbatim}

\begin{Shaded}
\begin{Highlighting}[]
\NormalTok{ECOTOX_Neonic <-}\StringTok{ }\KeywordTok{read.csv}\NormalTok{(}\StringTok{"./Data/Raw/ECOTOX_Neonicotinoids_Mortality_raw.csv"}\NormalTok{, }\DataTypeTok{header =} \OtherTok{TRUE}\NormalTok{) }\CommentTok{#header = TRUE to make sure there's no spaces in column names}
\KeywordTok{library}\NormalTok{(readr)}
\NormalTok{NTL_LTER_Lake_ChemistryPhysics_Raw <-}\StringTok{ }\KeywordTok{read_csv}\NormalTok{(}\StringTok{"Data/Raw/NTL-LTER_Lake_ChemistryPhysics_Raw.csv"}\NormalTok{)}
\end{Highlighting}
\end{Shaded}

\begin{verbatim}
## Parsed with column specification:
## cols(
##   lakeid = col_character(),
##   lakename = col_character(),
##   year4 = col_double(),
##   daynum = col_double(),
##   sampledate = col_character(),
##   depth = col_double(),
##   temperature_C = col_double(),
##   dissolvedOxygen = col_double(),
##   irradianceWater = col_double(),
##   irradianceDeck = col_double(),
##   comments = col_logical()
## )
\end{verbatim}

\begin{verbatim}
## Warning: 368 parsing failures.
##   row      col           expected                            actual                                              file
## 36649 comments 1/0/T/F/TRUE/FALSE DO Probe bad - Doesn't go to zero 'Data/Raw/NTL-LTER_Lake_ChemistryPhysics_Raw.csv'
## 36651 comments 1/0/T/F/TRUE/FALSE DO Probe bad - Doesn't go to zero 'Data/Raw/NTL-LTER_Lake_ChemistryPhysics_Raw.csv'
## 36653 comments 1/0/T/F/TRUE/FALSE DO Probe bad - Doesn't go to zero 'Data/Raw/NTL-LTER_Lake_ChemistryPhysics_Raw.csv'
## 36654 comments 1/0/T/F/TRUE/FALSE DO Probe bad - Doesn't go to zero 'Data/Raw/NTL-LTER_Lake_ChemistryPhysics_Raw.csv'
## 36655 comments 1/0/T/F/TRUE/FALSE DO Probe bad - Doesn't go to zero 'Data/Raw/NTL-LTER_Lake_ChemistryPhysics_Raw.csv'
## ..... ........ .................. ................................. .................................................
## See problems(...) for more details.
\end{verbatim}

\begin{Shaded}
\begin{Highlighting}[]
\CommentTok{#NTL-LTER_Lake_ChemistryPhysics_Raw <- read.csv("./Data/Raw/NTL-LTER_Lake_ChemistryPhysics_Raw.csv")}

\CommentTok{#2}
\NormalTok{A6theme <-}\StringTok{ }\KeywordTok{theme_gray}\NormalTok{(}\DataTypeTok{base_size =} \DecValTok{13}\NormalTok{)}
\KeywordTok{theme}\NormalTok{(}\DataTypeTok{axis.text =} \KeywordTok{element_text}\NormalTok{(}\DataTypeTok{color =} \StringTok{"black"}\NormalTok{), }\DataTypeTok{legend.position =} \StringTok{"right"}\NormalTok{)}
\end{Highlighting}
\end{Shaded}

\begin{verbatim}
## List of 2
##  $ axis.text      :List of 11
##   ..$ family       : NULL
##   ..$ face         : NULL
##   ..$ colour       : chr "black"
##   ..$ size         : NULL
##   ..$ hjust        : NULL
##   ..$ vjust        : NULL
##   ..$ angle        : NULL
##   ..$ lineheight   : NULL
##   ..$ margin       : NULL
##   ..$ debug        : NULL
##   ..$ inherit.blank: logi FALSE
##   ..- attr(*, "class")= chr [1:2] "element_text" "element"
##  $ legend.position: chr "right"
##  - attr(*, "class")= chr [1:2] "theme" "gg"
##  - attr(*, "complete")= logi FALSE
##  - attr(*, "validate")= logi TRUE
\end{verbatim}

\begin{Shaded}
\begin{Highlighting}[]
\KeywordTok{theme_set}\NormalTok{(A6theme)}
\end{Highlighting}
\end{Shaded}

\subsection{Neonicotinoids test}\label{neonicotinoids-test}

Research question: Were studies on various neonicotinoid chemicals
conducted in different years?

\begin{enumerate}
\def\labelenumi{\arabic{enumi}.}
\setcounter{enumi}{2}
\item
  Generate a line of code to determine how many different chemicals are
  listed in the Chemical.Name column.
\item
  Are the publication years associated with each chemical
  well-approximated by a normal distribution? Run the appropriate test
  and also generate a frequency polygon to illustrate the distribution
  of counts for each year, divided by chemical name. Bonus points if you
  can generate the results of your test from a pipe function. No need to
  make this graph pretty.
\item
  Is there equal variance among the publication years for each chemical?
  Hint: var.test is not the correct function.
\end{enumerate}

\begin{Shaded}
\begin{Highlighting}[]
\CommentTok{#3 how many chemicals are listed }
\KeywordTok{summary}\NormalTok{(ECOTOX_Neonic}\OperatorTok{$}\NormalTok{Chemical.Name) }\CommentTok{#(9 chemicals)}
\end{Highlighting}
\end{Shaded}

\begin{verbatim}
##  Acetamiprid Clothianidin  Dinotefuran Imidacloprid Imidaclothiz 
##          136           74           59          695            9 
##   Nitenpyram   Nithiazine  Thiacloprid Thiamethoxam 
##           21           22          106          161
\end{verbatim}

\begin{Shaded}
\begin{Highlighting}[]
\KeywordTok{class}\NormalTok{(ECOTOX_Neonic}\OperatorTok{$}\NormalTok{Pub..Year) }\CommentTok{#integer}
\end{Highlighting}
\end{Shaded}

\begin{verbatim}
## [1] "integer"
\end{verbatim}

\begin{Shaded}
\begin{Highlighting}[]
\CommentTok{#4  see if it's a normal distr }
\CommentTok{#not numeric, so need other test like ANOVA}

\NormalTok{Chem.Name <-}\StringTok{ }\ControlFlowTok{function}\NormalTok{(N) \{ECOTOX_Neonic }\OperatorTok
\StringTok{    }\KeywordTok{filter}\NormalTok{(Chemical.Name }\OperatorTok{==}\StringTok{ 'Acetamiprid'}\NormalTok{) }\OperatorTok
\StringTok{    }\KeywordTok{pull}\NormalTok{(Pub..Year) }\OperatorTok
\StringTok{    }\KeywordTok{shapiro.test}\NormalTok{()}
\NormalTok{\}}

\NormalTok{Chem.Name}
\end{Highlighting}
\end{Shaded}

\begin{verbatim}
## function(N) {ECOTOX_Neonic %>%
##     filter(Chemical.Name == 'Acetamiprid') %>%
##     pull(Pub..Year) %>%
##     shapiro.test()
## }
\end{verbatim}

\begin{Shaded}
\begin{Highlighting}[]
\NormalTok{Ecotox.PubYr.norm <-}\StringTok{ }\KeywordTok{ggplot}\NormalTok{(ECOTOX_Neonic) }\OperatorTok{+}
\KeywordTok{geom_freqpoly}\NormalTok{(}\KeywordTok{aes}\NormalTok{(}\DataTypeTok{x =}\NormalTok{ Pub..Year, }\DataTypeTok{color =}\NormalTok{ Chemical.Name), }\DataTypeTok{stat =} \StringTok{"count"}\NormalTok{) }\OperatorTok{+}
\StringTok{  }\KeywordTok{labs}\NormalTok{(}\DataTypeTok{x =} \StringTok{"year of publication"}\NormalTok{, }\DataTypeTok{y =} \StringTok{"number of publications"}\NormalTok{)}
\KeywordTok{print}\NormalTok{(Ecotox.PubYr.norm)}
\end{Highlighting}
\end{Shaded}

\includegraphics{Muzzy_A06_GLMs_files/figure-latex/studies in different years-1.pdf}

\begin{Shaded}
\begin{Highlighting}[]
\CommentTok{#5 equal var in pub yrs for each chemical?}

\KeywordTok{bartlett.test}\NormalTok{(ECOTOX_Neonic}\OperatorTok{$}\NormalTok{Pub..Year }\OperatorTok{~}\StringTok{ }\NormalTok{ECOTOX_Neonic}\OperatorTok{$}\NormalTok{Chemical.Name) }
\end{Highlighting}
\end{Shaded}

\begin{verbatim}
## 
##  Bartlett test of homogeneity of variances
## 
## data:  ECOTOX_Neonic$Pub..Year by ECOTOX_Neonic$Chemical.Name
## Bartlett's K-squared = 139.59, df = 8, p-value < 2.2e-16
\end{verbatim}

\begin{Shaded}
\begin{Highlighting}[]
\CommentTok{#Bartlett's K-squared = 139.59, df = 8, p-value < 2.2e-16}

\CommentTok{# p <0.0001, so we can reject the null; the variance is not the same for all the chemicals.}
\end{Highlighting}
\end{Shaded}

\begin{enumerate}
\def\labelenumi{\arabic{enumi}.}
\setcounter{enumi}{5}
\tightlist
\item
  Based on your results, which test would you choose to run to answer
  your research question?
\end{enumerate}

\begin{quote}
ANSWER: ``Were studies on various neonicotinoid chemicals conducted in
different years?'' Kruskal-Wallis test, because it compares multiple
groups and it's nonparametric.
\end{quote}

\begin{enumerate}
\def\labelenumi{\arabic{enumi}.}
\setcounter{enumi}{6}
\item
  Run this test below.
\item
  Generate a boxplot representing the range of publication years for
  each chemical. Adjust your graph to make it pretty.
\end{enumerate}

\begin{Shaded}
\begin{Highlighting}[]
\CommentTok{#7 test for studies of dif chemicals performed in dif years}
\CommentTok{#response ~ explanatory}
\KeywordTok{range}\NormalTok{(ECOTOX_Neonic}\OperatorTok{$}\NormalTok{Pub..Year)}
\end{Highlighting}
\end{Shaded}

\begin{verbatim}
## [1] 1982 2018
\end{verbatim}

\begin{Shaded}
\begin{Highlighting}[]
\KeywordTok{summary}\NormalTok{(ECOTOX_Neonic}\OperatorTok{$}\NormalTok{Chemical.Name)}
\end{Highlighting}
\end{Shaded}

\begin{verbatim}
##  Acetamiprid Clothianidin  Dinotefuran Imidacloprid Imidaclothiz 
##          136           74           59          695            9 
##   Nitenpyram   Nithiazine  Thiacloprid Thiamethoxam 
##           21           22          106          161
\end{verbatim}

\begin{Shaded}
\begin{Highlighting}[]
\NormalTok{Chem.PubYr.kruskal <-}\StringTok{ }\KeywordTok{kruskal.test}\NormalTok{(Pub..Year }\OperatorTok{~}\StringTok{ }\NormalTok{Chemical.Name, ECOTOX_Neonic)}
\NormalTok{Chem.PubYr.kruskal }\CommentTok{#Kruskal-Wallis chi-squared = 134.15, df = 8, p-value < 2.2e-16}
\end{Highlighting}
\end{Shaded}

\begin{verbatim}
## 
##  Kruskal-Wallis rank sum test
## 
## data:  Pub..Year by Chemical.Name
## Kruskal-Wallis chi-squared = 134.15, df = 8, p-value < 2.2e-16
\end{verbatim}

\begin{Shaded}
\begin{Highlighting}[]
\CommentTok{#8 boxplot of range of pub years for each chemical}
\CommentTok{#not informative enough: need better x axis, can't figure out units or numbers}
\NormalTok{Ecotox.PubYr.Chemicals <-}\StringTok{ }\KeywordTok{ggplot}\NormalTok{(ECOTOX_Neonic, }\KeywordTok{aes}\NormalTok{(}\DataTypeTok{stat =} \StringTok{"count"}\NormalTok{, }\DataTypeTok{y =}\NormalTok{ Pub..Year )) }\OperatorTok{+}
\KeywordTok{geom_boxplot}\NormalTok{(}\KeywordTok{aes}\NormalTok{(}\DataTypeTok{fill =}\NormalTok{ Chemical.Name), }\DataTypeTok{position =} \StringTok{"dodge"}\NormalTok{) }\OperatorTok{+}\StringTok{ }
\CommentTok{#labs(x = "number of publications", y = "publication year", title = "Publications on Neonicotinoids, 1982-2018") +}
\KeywordTok{theme}\NormalTok{(}\DataTypeTok{legend.position =} \StringTok{"right"}\NormalTok{)}
\KeywordTok{print}\NormalTok{(Ecotox.PubYr.Chemicals)}
\end{Highlighting}
\end{Shaded}

\includegraphics{Muzzy_A06_GLMs_files/figure-latex/pub years for chemicals-1.pdf}

\begin{enumerate}
\def\labelenumi{\arabic{enumi}.}
\setcounter{enumi}{8}
\tightlist
\item
  How would you summarize the conclusion of your analysis? Include a
  sentence summarizing your findings and include the results of your
  test in parentheses at the end of the sentence.
\end{enumerate}

\begin{quote}
ANSWER: Used Kruskal test, p-val \textless{}0.05, indicating significant
difference between the amount of publications for the different
chemicals. (results: Kruskal-Wallis chi-squared = 134.15, df = 8,
p-value \textless{} 2.2e-16)
\end{quote}

\subsection{NTL-LTER test}\label{ntl-lter-test}

Research question: What is the best set of predictors for lake
temperatures in July across the monitoring period at the North Temperate
Lakes LTER?

\begin{enumerate}
\def\labelenumi{\arabic{enumi}.}
\setcounter{enumi}{10}
\tightlist
\item
  Wrangle your NTL-LTER dataset with a pipe function so that it contains
  only the following criteria:
\end{enumerate}

\begin{itemize}
\tightlist
\item
  Only dates in July (hint: use the daynum column). No need to consider
  leap years.
\item
  Only the columns: lakename, year4, daynum, depth, temperature\_C
\item
  Only complete cases (i.e., remove NAs)
\end{itemize}

\begin{enumerate}
\def\labelenumi{\arabic{enumi}.}
\setcounter{enumi}{11}
\tightlist
\item
  Run an AIC to determine what set of explanatory variables (year4,
  daynum, depth) is best suited to predict temperature. Run a multiple
  regression on the recommended set of variables.
\end{enumerate}

\begin{Shaded}
\begin{Highlighting}[]
\CommentTok{#11 : dates in July: lakename, year4, daynum, depth, temperature_C, remove NAs (na.omit but only after pipe)}
\NormalTok{Lake.July.temps <-}\StringTok{ }\NormalTok{NTL_LTER_Lake_ChemistryPhysics_Raw }\OperatorTok
\KeywordTok{filter}\NormalTok{(daynum }\OperatorTok{>=}\StringTok{ }\DecValTok{182} \OperatorTok{&}\StringTok{ }\NormalTok{daynum }\OperatorTok{<=}\StringTok{ }\DecValTok{212}\NormalTok{) }\OperatorTok
\KeywordTok{select}\NormalTok{(lakename, year4, daynum, depth, temperature_C) }\OperatorTok
\KeywordTok{na.omit}\NormalTok{()}

\CommentTok{#12 AIC}

\CommentTok{#Correlations close to -1 represent strong negative correlations, correlations close to zero represent weak correlations, and correlations close to 1 represent strong positive correlations. The **R-squared value** is the correlation squared, becoming a number between 0 and 1. The R-squared value describes the percent of variance accounted for by the explanatory variables.}
\NormalTok{Lake.July.temps.AIC <-}\StringTok{ }\KeywordTok{lm}\NormalTok{(}\DataTypeTok{data =}\NormalTok{ Lake.July.temps, temperature_C }\OperatorTok{~}\StringTok{ }\NormalTok{depth }\OperatorTok{+}\StringTok{ }\NormalTok{daynum }\OperatorTok{+}\StringTok{ }\NormalTok{year4)}
\KeywordTok{step}\NormalTok{(Lake.July.temps.AIC)}
\end{Highlighting}
\end{Shaded}

\begin{verbatim}
## Start:  AIC=26016.31
## temperature_C ~ depth + daynum + year4
## 
##          Df Sum of Sq    RSS   AIC
## <none>                141118 26016
## - year4   1        80 141198 26020
## - daynum  1      1333 142450 26106
## - depth   1    403925 545042 39151
\end{verbatim}

\begin{verbatim}
## 
## Call:
## lm(formula = temperature_C ~ depth + daynum + year4, data = Lake.July.temps)
## 
## Coefficients:
## (Intercept)        depth       daynum        year4  
##    -6.45556     -1.94726      0.04134      0.01013
\end{verbatim}

\begin{Shaded}
\begin{Highlighting}[]
\NormalTok{Lake.July.temps.model <-}\StringTok{ }\KeywordTok{lm}\NormalTok{(}\DataTypeTok{data =}\NormalTok{ Lake.July.temps, temperature_C }\OperatorTok{~}\StringTok{ }\NormalTok{year4 }\OperatorTok{+}\StringTok{ }\NormalTok{daynum)}
\KeywordTok{step}\NormalTok{(Lake.July.temps.model)}
\end{Highlighting}
\end{Shaded}

\begin{verbatim}
## Start:  AIC=39151.36
## temperature_C ~ year4 + daynum
## 
##          Df Sum of Sq    RSS   AIC
## - year4   1      3.33 545046 39149
## <none>                545042 39151
## - daynum  1   1355.90 546398 39174
## 
## Step:  AIC=39149.42
## temperature_C ~ daynum
## 
##          Df Sum of Sq    RSS   AIC
## <none>                545046 39149
## - daynum  1    1356.6 546402 39172
\end{verbatim}

\begin{verbatim}
## 
## Call:
## lm(formula = temperature_C ~ daynum, data = Lake.July.temps)
## 
## Coefficients:
## (Intercept)       daynum  
##      4.4786       0.0417
\end{verbatim}

\begin{Shaded}
\begin{Highlighting}[]
\KeywordTok{summary}\NormalTok{(Lake.July.temps.model) }\CommentTok{#Residual standard error: 7.489 on 9719 degrees of freedom Multiple R-squared:  0.002489,    Adjusted R-squared:  0.002284 F-statistic: 12.13 on 2 and 9719 DF,  p-value: 5.503e-06}
\end{Highlighting}
\end{Shaded}

\begin{verbatim}
## 
## Call:
## lm(formula = temperature_C ~ year4 + daynum, data = Lake.July.temps)
## 
## Residuals:
##     Min      1Q  Median      3Q     Max 
## -12.289  -7.138  -2.601   8.061  21.408 
## 
## Coefficients:
##              Estimate Std. Error t value Pr(>|t|)    
## (Intercept)  0.363637  16.976619   0.021    0.983    
## year4        0.002060   0.008456   0.244    0.808    
## daynum       0.041693   0.008479   4.917 8.93e-07 ***
## ---
## Signif. codes:  0 '***' 0.001 '**' 0.01 '*' 0.05 '.' 0.1 ' ' 1
## 
## Residual standard error: 7.489 on 9719 degrees of freedom
## Multiple R-squared:  0.002489,   Adjusted R-squared:  0.002284 
## F-statistic: 12.13 on 2 and 9719 DF,  p-value: 5.503e-06
\end{verbatim}

\begin{Shaded}
\begin{Highlighting}[]
\CommentTok{#weak correlation: only 0.2% of variance is accounted for by explan var}

\NormalTok{Lake.July.temps.regression <-}\StringTok{ }\KeywordTok{lm}\NormalTok{(}\DataTypeTok{data =}\NormalTok{ Lake.July.temps, temperature_C }\OperatorTok{~}\StringTok{ }\NormalTok{year4 }\OperatorTok{+}\StringTok{ }\NormalTok{daynum)}
\KeywordTok{summary}\NormalTok{(Lake.July.temps.regression)}
\end{Highlighting}
\end{Shaded}

\begin{verbatim}
## 
## Call:
## lm(formula = temperature_C ~ year4 + daynum, data = Lake.July.temps)
## 
## Residuals:
##     Min      1Q  Median      3Q     Max 
## -12.289  -7.138  -2.601   8.061  21.408 
## 
## Coefficients:
##              Estimate Std. Error t value Pr(>|t|)    
## (Intercept)  0.363637  16.976619   0.021    0.983    
## year4        0.002060   0.008456   0.244    0.808    
## daynum       0.041693   0.008479   4.917 8.93e-07 ***
## ---
## Signif. codes:  0 '***' 0.001 '**' 0.01 '*' 0.05 '.' 0.1 ' ' 1
## 
## Residual standard error: 7.489 on 9719 degrees of freedom
## Multiple R-squared:  0.002489,   Adjusted R-squared:  0.002284 
## F-statistic: 12.13 on 2 and 9719 DF,  p-value: 5.503e-06
\end{verbatim}

\begin{Shaded}
\begin{Highlighting}[]
\NormalTok{Lake.July.temps.plot1 <-}\StringTok{ }\KeywordTok{ggplot}\NormalTok{(Lake.July.temps, }
                 \KeywordTok{aes}\NormalTok{(}\DataTypeTok{x =}\NormalTok{ temperature_C, }\DataTypeTok{y =}\NormalTok{ year4, }\DataTypeTok{color =}\NormalTok{ daynum)) }\OperatorTok{+}
\StringTok{  }\KeywordTok{geom_point}\NormalTok{(}\DataTypeTok{size =} \DecValTok{1}\NormalTok{) }
\KeywordTok{print}\NormalTok{(Lake.July.temps.plot1)}
\end{Highlighting}
\end{Shaded}

\includegraphics{Muzzy_A06_GLMs_files/figure-latex/unnamed-chunk-1-1.pdf}

\begin{Shaded}
\begin{Highlighting}[]
\NormalTok{Lake.July.temps.plot2 <-}\StringTok{ }\KeywordTok{ggplot}\NormalTok{(Lake.July.temps, }
                 \KeywordTok{aes}\NormalTok{(}\DataTypeTok{x =}\NormalTok{ daynum, }\DataTypeTok{y =}\NormalTok{ temperature_C, }\DataTypeTok{color =}\NormalTok{ year4)) }\OperatorTok{+}
\StringTok{  }\KeywordTok{geom_point}\NormalTok{(}\DataTypeTok{size =} \DecValTok{1}\NormalTok{)}
\KeywordTok{print}\NormalTok{(Lake.July.temps.plot2)}
\end{Highlighting}
\end{Shaded}

\includegraphics{Muzzy_A06_GLMs_files/figure-latex/unnamed-chunk-1-2.pdf}

\begin{enumerate}
\def\labelenumi{\arabic{enumi}.}
\setcounter{enumi}{12}
\tightlist
\item
  What is the final linear equation to predict temperature from your
  multiple regression? How much of the observed variance does this model
  explain?
\end{enumerate}

\begin{quote}
ANSWER: temperature\_C = 0.36 + 0.002(year4) + 0.04(daynum) + 16.9(E).
This model only explains 0.2\% of the variance, which is terrible.
\end{quote}

\begin{enumerate}
\def\labelenumi{\arabic{enumi}.}
\setcounter{enumi}{13}
\tightlist
\item
  Run an interaction effects ANCOVA to predict temperature based on
  depth and lakename from the same wrangled dataset.
\end{enumerate}

\begin{Shaded}
\begin{Highlighting}[]
\CommentTok{#14 lm}

\NormalTok{Lake.July.temps.ancova <-}\StringTok{ }\KeywordTok{lm}\NormalTok{(}\DataTypeTok{data =}\NormalTok{ Lake.July.temps, temperature_C }\OperatorTok{~}\StringTok{ }\NormalTok{lakename }\OperatorTok{+}\StringTok{ }\NormalTok{depth)}
\KeywordTok{summary}\NormalTok{(Lake.July.temps.ancova)}
\end{Highlighting}
\end{Shaded}

\begin{verbatim}
## 
## Call:
## lm(formula = temperature_C ~ lakename + depth, data = Lake.July.temps)
## 
## Residuals:
##     Min      1Q  Median      3Q     Max 
## -8.1127 -3.0040 -0.2316  2.8312 15.1985 
## 
## Coefficients:
##                          Estimate Std. Error  t value Pr(>|t|)    
## (Intercept)              21.68826    0.32512   66.709  < 2e-16 ***
## lakenameCrampton Lake     4.52447    0.38213   11.840  < 2e-16 ***
## lakenameEast Long Lake   -1.45418    0.34530   -4.211 2.56e-05 ***
## lakenameHummingbird Lake -4.88905    0.46179  -10.587  < 2e-16 ***
## lakenamePaul Lake         0.91157    0.33264    2.740  0.00615 ** 
## lakenamePeter Lake        1.37937    0.33250    4.148 3.38e-05 ***
## lakenameTuesday Lake     -1.42651    0.33815   -4.219 2.48e-05 ***
## lakenameWard Lake        -0.68248    0.46187   -1.478  0.13954    
## lakenameWest Long Lake   -0.20353    0.34392   -0.592  0.55400    
## depth                    -1.96627    0.01095 -179.552  < 2e-16 ***
## ---
## Signif. codes:  0 '***' 0.001 '**' 0.01 '*' 0.05 '.' 0.1 ' ' 1
## 
## Residual standard error: 3.538 on 9712 degrees of freedom
## Multiple R-squared:  0.7775, Adjusted R-squared:  0.7773 
## F-statistic:  3770 on 9 and 9712 DF,  p-value: < 2.2e-16
\end{verbatim}

\begin{enumerate}
\def\labelenumi{\arabic{enumi}.}
\setcounter{enumi}{14}
\tightlist
\item
  Is there an interaction between depth and lakename? How much variance
  in the temperature observations does this explain?
\end{enumerate}

\begin{quote}
ANSWER: There appears to be an interaction between depth and lakename
(which makes sense: lakes are probably going to have different depths).
It explains about 78\% of the variance.
\end{quote}

\begin{enumerate}
\def\labelenumi{\arabic{enumi}.}
\setcounter{enumi}{15}
\tightlist
\item
  Create a graph that depicts temperature by depth, with a separate
  color for each lake. Add a geom\_smooth (method = ``lm'', se = FALSE)
  for each lake. Make your points 50 \% transparent. Adjust your y axis
  limits to go from 0 to 35 degrees. Clean up your graph to make it
  pretty.
\end{enumerate}

\begin{Shaded}
\begin{Highlighting}[]
\CommentTok{#16 x=depth y=temp ?}

\NormalTok{Lakes.Temp.by.depth <-}\StringTok{ }\KeywordTok{ggplot}\NormalTok{(Lake.July.temps, }\KeywordTok{aes}\NormalTok{(}\DataTypeTok{x =}\NormalTok{ depth, }\DataTypeTok{y =}\NormalTok{ temperature_C), }\DataTypeTok{color =}\NormalTok{ depth) }\OperatorTok{+}
\StringTok{  }\KeywordTok{theme_bw}\NormalTok{() }\OperatorTok{+}
\StringTok{  }\KeywordTok{geom_point}\NormalTok{(}\DataTypeTok{alpha =} \FloatTok{0.5}\NormalTok{, }\DataTypeTok{size =} \FloatTok{0.2}\NormalTok{, }\DataTypeTok{color =} \StringTok{"gray"}\NormalTok{) }\OperatorTok{+}
\StringTok{  }\KeywordTok{ylim}\NormalTok{(}\DecValTok{0}\NormalTok{,}\DecValTok{35}\NormalTok{) }\OperatorTok{+}
\StringTok{  }\KeywordTok{geom_smooth}\NormalTok{(}\KeywordTok{aes}\NormalTok{(}\DataTypeTok{color =}\NormalTok{ lakename), }\DataTypeTok{method =} \StringTok{"lm"}\NormalTok{, }\DataTypeTok{se =} \OtherTok{FALSE}\NormalTok{, }\DataTypeTok{size =} \FloatTok{0.5}\NormalTok{) }\OperatorTok{+}
\KeywordTok{labs}\NormalTok{(}\DataTypeTok{x =} \StringTok{"Depth"}\NormalTok{, }\DataTypeTok{y =} \StringTok{"Temperature"}\NormalTok{, }\DataTypeTok{title =} \StringTok{"Lake Temperatures by Depth"}\NormalTok{) }
\KeywordTok{print}\NormalTok{(Lakes.Temp.by.depth)}
\end{Highlighting}
\end{Shaded}

\begin{verbatim}
## Warning: Removed 73 rows containing missing values (geom_smooth).
\end{verbatim}

\includegraphics{Muzzy_A06_GLMs_files/figure-latex/unnamed-chunk-3-1.pdf}


\end{document}
